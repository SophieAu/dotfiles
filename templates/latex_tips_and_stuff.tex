% listing (code) template
\begin{lstlisting}[caption={}, label={action}, language=]
\end{lstlisting}

% import source code from a file
\lstinputlisting{filename.java}[caption={}, label{action}, language=]

% to get a frame around the source code
\begin{framed}...\end{framed}



% figure (image) template (centered image, at the position that's specified in the .tex document ([H]))
\begin{figure}[H]
\begin{center}
	\includegraphics[width=\textwidth]{filename}
	\caption{}
	\label{}
\end{center}
\end{figure}

% right-aligned image with text wrapping around it
\begin{wrapfigure}{r}{0.4\textwidth} % float-side, width of float box
	\vspace{-35pt} % padding at the top
	\begin{center}
		\includegraphics[width=\textwidth]{filename}
	\end{center}
	\vspace{-10pt} % padding at the bottom
\end{wrapfigure}



%referencing a label:
\ref{}

%referencing a bibtex item:
\cite{}



% tables
\bgroup
\def\arraystretch{1.5}
\begin{tabular*}{\textwidth}{m{0.25\textwidth}|m{0.70\textwidth}}
	A & B\\
	\hline C & D\\
\end{tabular*} 
\egroup

\multicolumn{3}{c||}{<Text>}  % Text über 3 Spalten; zentriert (c) gefolgt von Zwei Linien (||)
\multirow{7}{2.9cm}{<Text>}  % Text über 7 Zeilen, 2.9cm breit, längerer Text wird umgebrochen
\multirow{7}{*}{<Text>}  % Text über 7 Zeilen, Breite wie im Header, längerer Text fließt aus der Spalte raus

% Text drehen
\parbox[c]{2mm}{<höhe>}
% 1. Erstellt eine Parbox, in der der Text zentriert ist. Die Box ist 2mm breit und <höhe> hoch
 höhe = \multirow{7}{*}{<text>} 
% 2. Erstellt eine MultiRow über 7 Zeilen, die so breit ist wie der Header angibt 
 text = \rotatebox[origin=c]{90}{<Finaler Text>}
% 3. Der Text wird um den Mittelpunkt (c) 90 Grad gedreht

